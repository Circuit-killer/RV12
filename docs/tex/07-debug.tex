\chapter{Debug Unit}\label{debug-unit}

\section{Introduction}\label{introduction-4}

The Debug Unit is a separate unit in the CPU. It's not directly related
to any instruction execution or support functions, like Cache or Branch
Prediction. Instead it provides a means to halt the CPU and inspect its
internal registers and state as a means of debugging the execution
program.

The Debug Unit has its own interfaces and must be connected to an
external debug controller that provides the actual interfacing to the
external Debug Tools. The Debug Unit does not stall the CPU, instead it
relies on the external debug controller to stall the CPU when the Debug
Unit requests it.

\section{Debug Controller Interface
}\label{debug-controller-interface}

The Debug Unit has two interfaces; one to communicate with the CPU and
one to communicate with the external debug controller. The CPU interface
is an internal interface and therefore not described here.

The Debug Controller Interface is an SRAM like synchronous interface.
The connected Debug Controller must use the same clock as the CPU.

\begin{longtable}[]{@{}lccl@{}}
\toprule
Port & Size & Direction & Description\tabularnewline
\midrule
\endhead
\texttt{dbg\_stall} & 1 & Input & Stall CPU\tabularnewline
\texttt{dbg\_strb}  & 1 & Input & Access Request/Strobe\tabularnewline
\texttt{dbg\_we}    & 1 & Input & Write Enable\tabularnewline
\texttt{dbg\_addr}  & 13 & Input & Address Bus\tabularnewline
\texttt{dbg\_dati}  & \texttt{XLEN} & Input & Write Data Bus\tabularnewline
\texttt{dbg\_dato}  & \texttt{XLEN} & Output & Read Data Bus\tabularnewline
\texttt{dbg\_ack}   & 1 & Output & Access Acknowledge\tabularnewline
\texttt{dbg\_bp}    & 1 & Output & BreakPoint\tabularnewline
\bottomrule
\caption{Debug Interface Signals}
\label{tab:debug-if-signals}
\end{longtable}

\subsection{dbg\_stall}\label{dbg_stall}

The CPU is halted when \texttt{dbg\_stall} is asserted (`1'). No new instructions
are fed into the execution units. Any instructions already issued are
finished.

The Debug Unit can use this signal to pause program execution and
inspect the CPU's state and registers. The Debug Controller must assert
\texttt{dbg\_stall} immediate (combinatorial) when the Debug Unit asserts
\texttt{dbg\_bp}.

\subsection{dbg\_strb}\label{dbg_strb}

The Debug Controller asserts (`1') the Access Strobe signal when it
wants to read from or write to the Debug Unit or the CPU's registers. It
must remain asserted until the Debug Unit acknowledges completion of the
access by asserting (`1') \texttt{dbg\_ack}.

\subsection{dbg\_we}\label{dbg_we}

The Debug Controller asserts (`1') the Write Enable signal when it wants
to write to the Debug Unit or the CPU's registers. It must remain
asserted until the Debug Unit acknowledges completion of the access by
asserting (`1') \texttt{dbg\_ack}. It is valid only when \texttt{dbg\_strb} is asserted as
well.

\subsection{dbg\_addr}\label{dbg_addr}

The address bus carries the register-address that is is read from or
written to. See Register Map for the details.

\subsection{dbg\_dati}\label{dbg_dati}

The write data bus carries the data to be written to the Debug Unit's or
CPU's registers.

\subsection{dbg\_dato}\label{dbg_dato}

The read data bus carries the data read from the Debug Unit's or CPU's
registers.

\subsection{dbg\_bp}\label{dbg_bp}

The Debug Unit asserts (`1') BreakPoint when a hardware breakpoint,
single-step, branch-trace, or exception hit occurred. This is the CPU
stall request from the Debug Unit to the external debug controller. The
Debug Controller must assert (`1') \texttt{dbg\_stall} immediately
(combinatorial) upon detecting \texttt{dbg\_bp} asserted.

\pagebreak

\section{Register Map}\label{register-map}

The Debug Unit's address map provides access to the Debug Unit's
internal registers, the Register Files, and the
Control-and-Status-Registers.

The internal registers can be always accessed, whereas the Register
Files and the CSRs can only be access when the CPU is stalled.

\begin{longtable}[]{@{}lcp{6cm}@{}}
	\toprule
	\textbf{addr[12:0]} & \textbf{Register} & \textbf{Description}\tabularnewline
	\midrule

\ifdefined\MARKDOWN
	\endhead
\else
	\endfirsthead
	\multicolumn{3}{c}{{(Continued from previous page)}} \\
	\toprule
	\textbf{addr{[}12:0{]}} & \textbf{Register} & \textbf{Description}\tabularnewline
	\midrule
	\endhead
	\midrule \multicolumn{3}{c}{{\tablename\ \thetable{} continued on next page\ldots}} \\
	\endfoot
	\endlastfoot

\fi

	0x0000        & \texttt{DBG\_CTRL}    & Debug Control\tabularnewline
	0x0001        & \texttt{DBG\_HIT}     & Debug Hit\tabularnewline
	0x0002        & \texttt{DBG\_IE}      & Debug Interrupt Enable\tabularnewline
	0x0003        & \texttt{DBG\_CAUSE}   & Debug Interrupt Cause\tabularnewline
	0x0004-0x000F &                       & \emph{Reserved}\tabularnewline
	0x0010        & \texttt{DBG\_BPCTRL0} & Hardware Breakpoint0 Control\tabularnewline
	0x0011        & \texttt{DBG\_BPDATA0} & Hardware Breakpoint0 Data\tabularnewline
	0x0012        & \texttt{DBG\_BPCTRL1} & Hardware Breakpoint1 Control\tabularnewline
	0x0013        & \texttt{DBG\_BPDATA1} & Hardware Breakpoint1 Data\tabularnewline
	0x0014        & \texttt{DBG\_BPCTRL2} & Hardware Breakpoint2 Control\tabularnewline
	0x0015        & \texttt{DBG\_BPDATA2} & Hardware Breakpoint2 Data\tabularnewline
	0x0016        & \texttt{DBG\_BPCTRL3} & Hardware Breakpoint3 Control\tabularnewline
	0x0017        & \texttt{DBG\_BPDATA3} & Hardware Breakpoint3 Data\tabularnewline
	0x0018        & \texttt{DBG\_BPCTRL4} & Hardware Breakpoint4 Control\tabularnewline
	0x0019        & \texttt{DBG\_BPDATA4} & Hardware Breakpoint4 Data\tabularnewline
	0x001A        & \texttt{DBG\_BPCTRL5} & Hardware Breakpoint5 Control\tabularnewline
	0x001B        & \texttt{DBG\_BPDATA5} & Hardware Breakpoint5 Data\tabularnewline
	0x001C        & \texttt{DBG\_BPCTRL6} & Hardware Breakpoint6 Control\tabularnewline
	0x001D        & \texttt{DBG\_BPDATA6} & Hardware Breakpoint6 Data\tabularnewline
	0x001E        & \texttt{DBG\_BPCTRL7} & Hardware Breakpoint7 Control\tabularnewline
	0x001F        & \texttt{DBG\_BPDATA7} & Hardware Breakpoint7 Data\tabularnewline
	0x0020-0x00FF &                       & \emph{Reserved}\tabularnewline
	0x0100-0x011F & \texttt{RF}           & Integer Register File\tabularnewline
	0x0120-0x03FF &                       & \emph{Reserved}\tabularnewline
	0x0140-0x051F & \texttt{FRF}          & Floating Point Register File\tabularnewline
	0x0160-0x071F & \texttt{FRF} (MSBs)   & MSBs of the Floating Point Register, for
	64bit \texttt{FRF} with 32bit \texttt{XLEN}\tabularnewline
	0x0180-0x07FF &                       & \emph{Reserved}\tabularnewline
	0x0800        & \texttt{NPC}          & Next Program Counter\tabularnewline
	0x0801        & \texttt{PPC}          & Current Program Counter\tabularnewline
	0x0802-0x0FFF &                       & \emph{Reserved}\tabularnewline
	0x1000-0x1FFF & CSR                   & CPU Control and Status\tabularnewline
	\bottomrule
	\caption{Debug Unit Register Map}
	\label{tab:debug-reg-map}
\end{longtable}

\section{Internal Register Map}\label{internal-register-map}

The Debug Unit's internal register map can be accessed when the CPU is
stalled or running. These registers control the hardware breakpoints and
conditions and report the reason why the Debug Unit stalled the CPU.

\subsection{Debug Control Register \texttt{DBG\_CTRL}} \label{debug-control-register-dbg_ctrl}

The \texttt{XLEN} size \texttt{DBG\_CTRL} controls the single-step and branch-tracing
functions.

\ifdefined\MARKDOWN
% Skip 
\else

\begin{figure*}[h!]
	{\footnotesize
		\begin{center}
			\begin{tabular}{@{}Ucc}
				\instbitrange{XLEN-1}{2} &
				\instbit{1} &
				\instbit{0} \\
				\hline
				\multicolumn{1}{|c|}{Reserved} &
				\multicolumn{1}{c|}{bte} &
				\multicolumn{1}{c|}{sste} \\
				\hline
				XLEN-2 & 1 & 1 \\
			\end{tabular}
		\end{center}
	}
	\vspace{-0.1in}
	\caption{Debug Control Register \texttt{DBG\_CTRL}.}
	\label{fig:dbgctrlreg}
\end{figure*}

\fi

When the Single-Step-Trace-Enable bit is `1' the Single-Step-Trace
function is enabled. The CPU will assert (`1') \texttt{dbg\_bp} each time a
non-NOP instruction is about to be executed.

\begin{longtable}[]{@{}cl@{}}
\toprule
\textbf{sste} & \textbf{Description}\tabularnewline
\midrule
\endhead
0 & Single-Step-Trace disabled\tabularnewline
1 & Single-Step-Trace enabled\tabularnewline
\bottomrule
\caption{Single Step Trace Enable Settings}
\label{tab:single-step-trace-settings}
\end{longtable}

When the Branch-Trace-Enable bit is `1' the Branch-Step-Trace function
is enabled. The CPU will assert \texttt{dbg\_bp} each time a branch instruction
is about to be executed.

\begin{longtable}[]{@{}cl@{}}
\toprule
\textbf{bte} & \textbf{Description}\tabularnewline
\midrule
\endhead
0 & Branch-Step-Trace disabled\tabularnewline
1 & Branch-Step-Trace enabled\tabularnewline
\bottomrule
\caption{Branch Trace Enable Settings}
\label{tab:branch-trace-settings}
\end{longtable}


\subsection{Debug Breakpoint Hit Register \texttt{DBG\_HIT}} \label{debug-breakpoint-hit-register-dbg_hit}

\ifdefined\MARKDOWN
% Skip 
\else

\begin{figure*}[h!]
	{\footnotesize
		\begin{center}
			\begin{tabular}{@{}YccccccccFcc}
				\instbitrange{XLEN-1}{16} &
				\instbit{15} &
				\instbit{14} &
				\instbit{13} &
				\instbit{12} &
				\instbit{11} &
				\instbit{10} &
				\instbit{9} &
				\instbit{8} &
				\instbitrange{7}{2} &
				\instbit{1} &
				\instbit{0} \\
				\hline
				\multicolumn{1}{|c|}{Reserved} &
				\multicolumn{1}{c|}{bp7h} &
				\multicolumn{1}{c|}{bp7h} &
				\multicolumn{1}{c|}{bp7h} &
				\multicolumn{1}{c|}{bp7h} &
				\multicolumn{1}{c|}{bp7h} &
				\multicolumn{1}{c|}{bp7h} &
				\multicolumn{1}{c|}{bp7h} &
				\multicolumn{1}{c|}{bp7h} &
				\multicolumn{1}{c|}{6`h0} &
				\multicolumn{1}{c|}{bth} &
				\multicolumn{1}{c|}{sste} \\
				\hline
				XLEN-16 & 1 & 1 & 1 & 1 & 1 & 1 & 1 & 1 & 6 & 1 & 1\\
				
			\end{tabular}
		\end{center}
	}
	\vspace{-0.1in}
	\caption{Debug Breakpoint Hit Register}
	\label{fig:dbghitreg}
\end{figure*}

\fi

The Debug Breakpoint Hit register contains the reason(s) why the Debug
Unit requested to stall the CPU.

The Single-Step-Trace-Hit field is asserted (`1') when the
Single-Step-Trace function requests to stall the CPU. This is a sticky
bit. It is set by the Debug Unit, but must be cleared by the Debug
Environment.

The Branch-Trace-Hit field is asserted (`1') when the Branch-Trace
function requests to stall the CPU. This is a sticky bit. It is set by
the Debug Unit, but must be cleared by the Debug Environment.

The Breakpoint-Hit fields are asserted (`1') when the respective
hardware breakpoint triggered and requests to stall the CPU. There is
one bit for each implemented hardware breakpoint. These are sticky bits.
They are set by the Debug Unit, but must be cleared by the Debug
Environment.

\subsection{Debug Interrupt Enable Register \texttt{DBG\_IE}} \label{debug-interrupt-enable-register-dbg_ie}

\ifdefined\MARKDOWN
% Skip 
\else

\begin{figure*}[h!]
	{\footnotesize
		\begin{center}
			\begin{tabular}{@{}U}
				\instbitrange{31}{0} \\
				\hline
				\multicolumn{1}{|c|}{ie}  \\
				\hline
			\end{tabular}
		\end{center}
	}
	\vspace{-0.1in}
	\caption{Debug Interrupt Enable Register \texttt{DBGIE}.}
	\label{fig:dbgiereg}
\end{figure*}

\fi

\begin{longtable}[]{@{}cl@{}}
	\toprule
	\textbf{Bit\#} & \textbf{Description}\tabularnewline
	\midrule

\ifdefined\MARKDOWN
	\endhead
\else
	\endfirsthead
	\multicolumn{2}{c}{{(Continued from previous page)}} \\
	\toprule
	\textbf{Bit\#} & \textbf{Description}\tabularnewline
	\midrule
	\endhead
	\midrule \multicolumn{2}{c}{{\tablename\ \thetable{} continued on next page\ldots}} \\
	\endfoot
	\endlastfoot

\fi
	31-18 & External Interrupts\tabularnewline
	17 & Timer Interrupt\tabularnewline
	16 & Software Interrupt\tabularnewline
	11 & Environment call from Machine Mode\tabularnewline
	10 & Environment call from Hypervisor Mode\tabularnewline
	9 & Environment call from Supervisor Mode\tabularnewline
	8 & Environment call from User Mode\tabularnewline
	7 & Store Access Fault\tabularnewline
	6 & Store Address Misaligned\tabularnewline
	5 & Load Access Fault\tabularnewline
	4 & Load Address Misaligned\tabularnewline
	3 & Breakpoint\tabularnewline
	2 & Illegal Instruction\tabularnewline
	1 & Instruction Access Fault\tabularnewline
	0 & Instruction Address Misaligned\tabularnewline
	\bottomrule
	\caption{\texttt{DBG\_IE} Register Bit Descriptions}
	\label{tab:dbg-ie-reg-bits}
\end{longtable}

The \texttt{dbg\_ie} register determines what exceptions cause the Debug Unit to
assert \texttt{dbg\_bp}. Normally an exception causes the CPU to load the
trap-vector and enter the trap routine, but if the applicable bit in the
\texttt{dbg\_ie} bit is set, then the CPU does not load the trap-vector, does not
change \texttt{mcause} and \texttt{mepc}, and does not enter the trap vector routine when
that exception is triggered. Instead the CPU sets \texttt{DBG\_CAUSE} and asserts
\texttt{dbg\_bp}, thereby handing over control to the external debug controller.

The lower 16bits of the register represent the trap causes as defined in
the \texttt{mcause} register. The upper 16bits represent the interrupt causes as
defined in the \texttt{mcause} register.

Logic `1' indicates the CPU hands over execution to the debug controller
when the corresponding exception is triggered. For example setting bit-2
to `1' causes the \texttt{BREAKPOINT} trap to assert \texttt{dbg\_bp} and hand over
control to the debug controller. At least the \texttt{BREAKPOINT} exception must
be set in the \texttt{dbg\_ie} register.

\subsection{Debug Exception Cause Register
\texttt{DBG\_CAUSE}}\label{debug-exception-cause-register-dbg_cause}

\ifdefined\MARKDOWN
% Skip 
\else

\begin{figure*}[h!]
	{\footnotesize
		\begin{center}
			\begin{tabular}{@{}U}
				\instbitrange{31}{0} \\
				\hline
				\multicolumn{1}{|c|}{cause}  \\
				\hline
			\end{tabular}
		\end{center}
	}
	\vspace{-0.1in}
	\caption{Debug Exception Cause Register \texttt{DBG\_CAUSE}.}
	\label{fig:dbgcausereg}
\end{figure*}

\fi

The \texttt{DBG\_CAUSE} register contains the exception number that caused the
CPU to hand over control to the external Debug Controller. See the
\texttt{mcause} register description for a description of all exceptions.

\begin{longtable}[]{@{}clc@{}}
	\toprule
	\textbf{DBG\_CAUSE} & \textbf{Description}           & \textbf{GDB Sigval}\tabularnewline
	\midrule
	\endhead
	\textgreater{}15    & Interrupts                     & INT\tabularnewline
	                    & Timer Interrupt                & ALRM\tabularnewline
	11                  & ECALL from Machine Mode        & TRAP\tabularnewline
	10                  & ECALL from Hypervisor Mode     & TRAP\tabularnewline
	9                   & ECALL from Supervisor Mode     & TRAP\tabularnewline
	8                   & ECALL from User Mode           & TRAP\tabularnewline
	7                   & Store Access Fault             & SEGV\tabularnewline
	6                   & Store Address Misaligned       & BUS\tabularnewline
	5                   & Load Access Fault              & SEGV\tabularnewline
	4                   & Load Address Misaligned        & BUS\tabularnewline
	3                   & Breakpoint                     & TRAP\tabularnewline
	2                   & Illegal Instruction            & ILL\tabularnewline
	1                   & Instruction Access Fault       & SEGV\tabularnewline
	0                   & Instruction Address Misaligned & BUS\tabularnewline
	\bottomrule
	\caption{DBG\_CAUSE Register Values}
\end{longtable}

Because the RISC-V defines the cause register as an integer value, there
is no easy way to detect if there was no cause. It's recommended that
the Debug Environment writes `-1' into the \texttt{dbg\_cause} register upon
starting the debug session and after handling each exception.

The debug controller's software layer must translate the value in the
\texttt{DBG\_CAUSE} register to the debugger's control signal. The table below
shows the basic mapping of the \texttt{DBG\_CAUSE} register to GDB Signals.

\subsection{Debug Breakpoint Control Registers
\texttt{DBG\_CTRLx}}\label{debug-breakpoint-control-registers-dbg_ctrlx}

\ifdefined\MARKDOWN
% Skip 
\else

\begin{figure*}[h!]
	{\footnotesize
		\begin{center}
			\begin{tabular}{@{}UFWcc}
				\instbitrange{31}{7} &
				\instbitrange{6}{4} &
				\instbitrange{3}{2} &
				\instbit{1} &
				\instbit{0} \\
				\hline
				\multicolumn{1}{|c|}{reserved} &
				\multicolumn{1}{|c|}{cc} &
				\multicolumn{1}{|c|}{00} &
				\multicolumn{1}{|c|}{ena} &
				\multicolumn{1}{|c|}{impl} \\
				\hline
				25 & 3 & 2 & 1 & 1\\
			\end{tabular}
		\end{center}
	}
	\vspace{-0.1in}
	\caption{Debug Breakpoint Control Registers \texttt{DBG\_CTRLx}.}
	\label{fig:dbgctrlxreg}
\end{figure*}

\fi

The \texttt{DBG\_BPCTRL} registers control the functionality of the hardware
breakpoints. There is a Breakpoint Control Register for each implemented
hardware breakpoint. The \texttt{BREAKPOINTS} parameter defines the amount of
hardware breakpoints that are implemented.

The Breakpoint Implemented field informs the Debug Environment if the
hardware breakpoint is implemented. The bit is set (`1') when the
hardware breakpoint is implemented and (`0') when it is not. The Debug
Environment should read the \texttt{DBG\_BPCTRL} registers and examine the
Breakpoint Implemented fields to determine the amount of hardware
breakpoints implemented.

\begin{longtable}[]{@{}cl@{}}
\toprule
impl & Description\tabularnewline
\midrule
\endhead
0 & Hardware Breakpoint not implemented\tabularnewline
1 & Hardware Breakpoint implemented\tabularnewline
\bottomrule
\caption{DBG\_CTRLx Implementation Field Values}
\end{longtable}


The Breakpoint Enable bit enables or disables the breakpoint. The
hardware breakpoint is enabled when the bit is set (`1') and disabled
when the bit is cleared (`0'). When the hardware breakpoint is disabled
it will not generate a breakpoint hit, even if the breakpoint conditions
are met. Clearing the breakpoint enable bit does not clear any pending
hits. These must be cleared in the \texttt{DBG\_HIT} register.

\begin{longtable}[]{@{}cl@{}}
\toprule
ena & Description\tabularnewline
\midrule
\endhead
0 & Hardware Breakpoint is disabled\tabularnewline
1 & Hardware Breakpoint is enabled\tabularnewline
\bottomrule
\caption{DBG\_CTRLx Enable Field Values}
\end{longtable}


The Breakpoint Condition Code bits determine what condition triggers the
hardware breakpoint.

\begin{longtable}[]{@{}cl@{}}
\toprule
cc & Description\tabularnewline
\midrule
\endhead
3'b000 & Instruction Fetch\tabularnewline
3'b001 & Data Load\tabularnewline
3'b010 & Data Store\tabularnewline
3'b011 & Data Access\tabularnewline
3'b1-\/- & Reserved\tabularnewline
\bottomrule
\caption{\texttt{DBG\_CTRLx} Breakpoint Condition Codes}
\end{longtable} 

\subsubsection{Instruction Fetch}\label{instruction-fetch}

The hardware breakpoint will trigger a breakpoint exception when the CPU
is about to execute the instruction at the address specified in the
\texttt{DBG\_DATA} register.

\subsubsection{Data Load}\label{data-load}

The hardware breakpoint will trigger a breakpoint exception when the CPU
reads from the address specified in the \texttt{DBG\_DATA} register.

\subsubsection{Data Store}\label{data-store}

The hardware breakpoint will trigger a breakpoint exception when the CPU
writes to the address specified in the \texttt{DBG\_DATA} register.

\subsubsection{Data Access}\label{data-access}

The hardware breakpoint will trigger a breakpoint exception when the CPU
accesses (either reads from or writes to) the address specified in the
\texttt{DBG\_DATA} register.

\subsection{Debug Breakpoint Data Registers
\texttt{DBG\_DATAx}}\label{debug-breakpoint-data-registers-dbg_datax}

\ifdefined\MARKDOWN
% Skip 
\else

\begin{figure*}[h!]
	{\footnotesize
		\begin{center}
			\begin{tabular}{@{}U}
				\instbitrange{XLEN-1}{0} \\
				\hline
				\multicolumn{1}{|c|}{data}  \\
				\hline
			\end{tabular}
		\end{center}
	}
	\vspace{-0.1in}
	\caption{Debug Breakpoint Data Registers \texttt{DBG\_DATA}.}
	\label{fig:dbgdatareg}
\end{figure*}

\fi

The \texttt{DBG\_DATA} registers contain the data/value that trigger a breakpoint
hit. There is a Breakpoint Data Register for each implemented hardware
breakpoint. The meaning of the \texttt{DBG\_DATA} register depends on the
condition code set in the associated \texttt{DBG\_BPCTRL} register. See the
\texttt{DBG\_CTRL} register for the meaning of the \texttt{DBG\_DATA} register.