\section{External Interfaces}\label{external-interfaces}

The RV12 CPU is designed to support a variety of external bus interfaces.
The following sections define the default AMBA3 AHB-Lite and Interrupt
Interfaces.

\subsection{AMBA3 AHB-Lite}\label{amba3-ahb-lite}

\begin{longtable}[]{@{}cccl@{}}
\toprule
Port & Size & Direction & Description\tabularnewline
\midrule
\endhead
\texttt{HRESETn} & 1 & Input & Asynchronous active low reset\tabularnewline
\texttt{HCLK} & 1 & Input & System clock input\tabularnewline
& & &\tabularnewline
\texttt{IHSEL} & 1 & Output & Provided for AHB-Lite compatibility -- tied high
(`1')\tabularnewline
\texttt{IHADDR} & \texttt{XLEN} & Output & Instruction address\tabularnewline
\texttt{IHRDATA} & 32 & Input & Instruction data\tabularnewline
\texttt{IHWRITE} & 1 & Output & Instruction write\tabularnewline
\texttt{IHSIZE} & 3 & Output & Transfer size\tabularnewline
\texttt{IHBURST} & 3 & Output & Transfer burst size\tabularnewline
\texttt{IHPROT} & 4 & Output & Transfer protection level\tabularnewline
\texttt{IHTRANS} & 2 & Output & Transfer type\tabularnewline
\texttt{IHMASTLOCK} & 1 & Output & Transfer master lock\tabularnewline
\texttt{IHREADY} & 1 & Input & Slave Ready Indicator\tabularnewline
\texttt{IHRESP} & 1 & Input & Instruction Transfer Response\tabularnewline
& & &\tabularnewline
\texttt{DHSEL} & 1 & Output & Provided for AHB-Lite compatibility -- tied high
(`1')\tabularnewline
\texttt{DHADDR} & \texttt{XLEN} & Output & Data address\tabularnewline
\texttt{DHRDATA} & \texttt{XLEN} & Input & Data read data\tabularnewline
\texttt{DHWDATA} & \texttt{XLEN} & Output & Data write data\tabularnewline
\texttt{DHWRITE} & 1 & Output & Instruction write\tabularnewline
\texttt{DHSIZE} & 3 & Output & Transfer size\tabularnewline
\texttt{DHBURST} & 3 & Output & Transfer burst size\tabularnewline
\texttt{DHPROT} & 4 & Output & Transfer protection level\tabularnewline
\texttt{DHTRANS} & 2 & Output & Transfer type\tabularnewline
\texttt{DHMASTLOCK} & 1 & Output & Transfer master lock\tabularnewline
\texttt{DHREADY} & 1 & Input & Slave Ready Indicator\tabularnewline
\texttt{DHRESP} & 1 & Input & Data Transfer Response\tabularnewline
& & &\tabularnewline
\bottomrule
\caption{AMBA3 AHB-Lite Ports}
\label{tab:ahb-ports}
\end{longtable}

\subsubsection{HRESETn}\label{hresetn}

When the active low asynchronous \texttt{HRESETn} input is asserted (`0'), the
core is put into its initial reset state.

\subsubsection{HCLK}\label{hclk}

\texttt{HCLK} is the system clock. All internal logic operates at the rising edge
of the system clock. All AHB bus timings are related to the rising edge
of \texttt{HCLK}.

\subsubsection{IHSEL}\label{ihsel}

\texttt{IHSEL} is a \emph{slave} selection signal and therefore provided for
AHB-Lite completeness. This signal is tied permanently high (`1')

\subsubsection{IHADDR}\label{ihaddr}

\texttt{IHADDR} is the instruction address bus. Its size is determined by
\texttt{PHYS\_ADDR\_SIZE}.

\subsubsection{IHRDATA}\label{ihrdata}

\texttt{IHRDATA} transfers the instruction from memory to the CPU. Its size is
determined by XLEN.
\texttt{
}\subsubsection{IHWRITE}\label{ihwrite}

\texttt{IHWRITE} indicates whether the current transfer is a read or a write
transfer. The instruction write is always negated (`0').

\subsubsection{IHSIZE}\label{ihsize}

The instruction transfer size is indicated by \texttt{IHSIZE}. Its value depends
on the \texttt{XLEN} parameter and if the current transfer is a cache-line fill
or non-cacheable instruction read.

\begin{longtable}[]{@{}ccp{7cm}@{}}
\toprule
\texttt{IHSIZE} & Type & Description\tabularnewline
\midrule
\endhead
\texttt{010} & Word & Non-cacheable instruction read. \texttt{XLEN=32}\tabularnewline
011 & Dword & Non-cacheable instruction read. \texttt{XLEN=64}\tabularnewline
\texttt{1-\/-} & & Cache line fill. The actual size depends on the Instruction
cache parameters and \texttt{XLEN}\tabularnewline
\bottomrule
\caption{Supported IHSIZE Values}
\label{tab:isize-values}
\end{longtable}

\subsubsection{IHBURST}\label{ihburst}

The instruction burst type indicates if the transfer is a single
transfer or part of a burst.

\begin{longtable}[]{@{}cll@{}}
\toprule
IHBURST & Type & Description\tabularnewline
\midrule
\endhead
\texttt{000} & Single & \emph{Not used}\tabularnewline
\texttt{001} & INCR & Non-cacheable instruction reads\tabularnewline
\texttt{010} & WRAP4 & 4-beat wrapping burst\tabularnewline
\texttt{011} & INCR4 & \emph{Not used}\tabularnewline
\texttt{100} & WRAP8 & 8-beat wrapping burst\tabularnewline
\texttt{101} & INCR8 & \emph{Not used}\tabularnewline
\texttt{110} & WRAP16 & 16-bear wrapping burst\tabularnewline
\texttt{111} & INCR16 & \emph{Not used}\tabularnewline
\bottomrule
\caption{Supported IHBURST Values}
\label{tab:ihburst-values}
\end{longtable}

\subsubsection{IHPROT}\label{ihprot}

The instruction protection signals provide information about the bus
transfer. They are intended to implement some level of protection.

\begin{longtable}[]{@{}ccl@{}}
\toprule
Bit\# & Value & Description\tabularnewline
\midrule
\endhead
3 & 1 & Cacheable region addressed\tabularnewline
& 0 & Non-cacheable region addressed\tabularnewline
2 & 1 & Bufferable\tabularnewline
& 0 & Non-bufferable\tabularnewline
1 & 1 & Privileged access. CPU is not in User Mode\tabularnewline
& 0 & User access. CPU is in User Mode\tabularnewline
0 & 0 & Opcode fetch, always `0'\tabularnewline
\bottomrule
\caption{Supported IHPROT Values}
\label{tab:ihprot-values}
\end{longtable}


\subsubsection{IHTRANS}\label{ihtrans}

\texttt{IHTRANS} indicates the type of the current instruction transfer.

\begin{longtable}[]{@{}cll@{}}
\toprule
IHTRANS & Type & Description\tabularnewline
\midrule
\endhead
\texttt{00} & IDLE & No transfer required\tabularnewline
\texttt{01} & BUSY & CPU inserts wait states during instruction burst read\tabularnewline
\texttt{10} & NONSEQ & First transfer of an instruction read burst\tabularnewline
\texttt{11} & SEQ & Remaining transfers of an instruction readburst\tabularnewline
\bottomrule
\caption{Supported IHTRANS Values}
\label{tab:ihtrans-values}
\end{longtable}

\subsubsection{IHMASTLOCK}\label{ihmastlock}

The instruction master lock signal indicates if the current transfer is
part of a locked sequence, commonly used for Read-Modify-Write cycles.
The instruction master lock is always negated (`0').

\subsubsection{IHREADY}\label{ihready}

\texttt{IHREADY} indicates whether the addressed slave is ready to transfer data
or not. When \texttt{IHREADY} is negated (`0') the slave is not ready, forcing
wait states. When \texttt{IHREADY} is asserted (`0') the slave is ready and the
transfer completed.

\subsubsection{IHRESP}\label{ihresp}

\texttt{IHRESP} is the instruction transfer response; it can either be \texttt{OKAY} (`0')
or \texttt{ERROR} (`1'). An error response causes a Bus Error exception.

\subsubsection{DHSEL}\label{dhsel}

\texttt{DHSEL} is a \emph{slave} selection signal and therefore provided for AHB-Lite
completeness. This signal is tied permanently high (`1')

\subsubsection{DHADDR}\label{dhaddr}

\texttt{DHADDR} is the data address bus. Its size is determined by
\texttt{PHYS\_ADDR\_SIZE}.

\subsubsection{DHRDATA}\label{dhrdata}

\texttt{DHRDATA} transfers the data from memory to the CPU. Its size is
determined by \texttt{XLEN}.

\subsubsection{DHWDATA}\label{dhwdata}

\texttt{DHWDATA} transfers the data from the CPU to memory. Its size is
determined by \texttt{XLEN}.

\subsubsection{DHWRITE}\label{dhwrite}

\texttt{DHWRITE} indicates whether the current transfer is a read or a write
transfer. It is asserted (`1') during a write and negated (`0') during a
read transfer.

\subsubsection{DHSIZE}\label{dhsize}

The data transfer size is indicated by DHSIZE. Its value depends on the
\texttt{XLEN} parameter and if the current transfer is a cache-line
fill/write-back or a non-cacheable data transfer.

\begin{longtable}[]{@{}clp{7.5cm}@{}}
\toprule
DHSIZE & Type & Description\tabularnewline
\midrule
\endhead
\texttt{000} & Byte & Non-cacheable data transfer\tabularnewline
\texttt{001} & Halfword & Non-cacheable data transfer\tabularnewline
\texttt{010} & Word & Non-cacheable data transfer\tabularnewline
\texttt{011} & Dword & Non-cacheable data transfer\tabularnewline
\texttt{1-\/-} & & Cache line fill. The actual size depends on the Instruction
cache parameters and \texttt{XLEN}\tabularnewline
\bottomrule
\caption{Supported DHSIZE Values}
\label{tab:dhsize-values}
\end{longtable}

\subsubsection{DHBURST}\label{dhburst}

The instruction burst type indicates if the transfer is a single
transfer or part of a burst.

\begin{longtable}[]{@{}ccl@{}}
\toprule
DHBURST & Type & Description\tabularnewline
\midrule
\endhead
000 & Single & Single transfer. E.g. non-cacheable
read/write\tabularnewline
001 & INCR & \emph{Not used}\tabularnewline
010 & WRAP4 & 4-beat wrapping burst\tabularnewline
011 & INCR4 & \emph{Not used}\tabularnewline
100 & WRAP8 & 8-beat wrapping burst\tabularnewline
101 & INCR8 & \emph{Not used}\tabularnewline
110 & WRAP16 & 16-bear wrapping burst\tabularnewline
111 & INCR16 & \emph{Not used}\tabularnewline
\bottomrule
\caption{Supported DHBURST Values}
\label{tab:dhburst-values}
\end{longtable}


\subsubsection{DHPROT}\label{dhprot}

The data protection signals provide information about the bus transfer.
They are intended to implement some level of protection.

\begin{longtable}[]{@{}ccl@{}}
\toprule
Bit\# & Value & Description\tabularnewline
\midrule
\endhead
3 & 1 & Cacheable region addressed\tabularnewline
& 0 & Non-cacheable region addressed\tabularnewline
2 & 1 & Bufferable\tabularnewline
& 0 & Non-bufferable\tabularnewline
1 & 1 & Privileged access. CPU is not in User Mode\tabularnewline
& 0 & User access. CPU is in User Mode\tabularnewline
0 & 1 & Data transfer, always `1'\tabularnewline
\bottomrule
\caption{Supported DHPROT Values}
\label{tab:dhprot-values}
\end{longtable}

\subsubsection{DHTRANS}\label{dhtrans}

\texttt{DHTRANS} indicates the type of the current data transfer.

\begin{longtable}[]{@{}ccl@{}}
\toprule
DHTRANS & Type & Description\tabularnewline
\midrule
\endhead
00 & IDLE & No transfer required\tabularnewline
01 & BUSY & \emph{Not used}\tabularnewline
10 & NONSEQ & First transfer of an data burst\tabularnewline
11 & SEQ & Remaining transfers of an data burst\tabularnewline
\bottomrule
\caption{Supported DHTRANS Values}
\label{tab:dhtrans-values}
\end{longtable}

\subsubsection{DHMASTLOCK}\label{dhmastlock}

The data master lock signal indicates if the current transfer is part of
a locked sequence, commonly used for Read-Modify-Write cycles. The data
master lock is always negated (`0').

\subsubsection{DHREADY}\label{dhready}

\texttt{DHREADY} indicates whether the addressed slave is ready to transfer data
or not. When \texttt{DHREADY} is negated (`0') the slave is not ready, forcing
wait states. When \texttt{DHREADY} is asserted (`0') the slave is ready and the
transfer completed.

\subsubsection{DHRESP}\label{dhresp}

\texttt{DHRESP} is the data transfer response; it can either be \texttt{OKAY} (`0') or
\texttt{ERROR} (`1'). An error response causes a Bus Error exception.

\subsection{Interrupts}\label{interrupts}

The RV12 supports multiple external interrupts and is designed to
operate in conjunction with an external Platform Level Interrupt
Controller (PLIC) as defined in Chapter 7 of the RISC-V Privilege Level
specification v1.9.1.

Dedicated pins on the RV12 core present the interrupt to the CPU which
then expects the Identifier of the Source Interrupt to be presented by
the PLIC at the appropriate interrupt vector upon a claim of the
interrupt.

\begin{longtable}[]{@{}cccl@{}}
\toprule
Port & Size & Direction & Description\tabularnewline
\midrule
\endhead
EXT\_NMI & 1 & Input & Non-Maskable Interrupt\tabularnewline
EXT\_TINT & 1 & Input & Timer Interrupt\tabularnewline
EXT\_SINT & 1 & Input & Software Interrupt\tabularnewline
EXT\_INT & 4 & Input & External Interrupts\tabularnewline
\bottomrule
\caption{Interrupts Supported}
\label{tab:int-support}
\end{longtable}

\subsubsection{EXT\_NMI}\label{ext_nmi}

The RV12 supports a single external non-maskable interrupt, accessible
in Machine Mode only. The interrupt vector for \texttt{EXT\_NMI} is defined as an
RV12 core parameter \texttt{MNMIVEC\_DEFAULT} (see section \ref{core-parameters} )

\subsubsection{EXT\_TINT}\label{ext_tint}

The RV12 supports a single Machine-Mode timer interrupt \texttt{EXT\_TINT}.

The interrupt may be delegated to other operating modes via software
manipulation of \texttt{mip} and \texttt{sip} registers (See sections 7.6.8 and 7.7.3).
Alternatively, higher performance interrupt redirection may be
implemented via use of the \texttt{mideleg} and \texttt{sideleg} configuration registers
(See sections \ref{machine-exception-interrupt-delegation-registers-medeleg-mideleg} and \ref{supervisor-trap-delegation-registers-sedeleg-sideleg} ).

The interrupt vector used to service the interrupt is determined based
on the mode the interrupt is delegated to via the \texttt{MTVEC\_DEFAULT},
\texttt{STVEC\_DEFAULT} and \texttt{UTVEC\_DEFAULT} parameters.

\subsubsection{EXT\_SINT}\label{ext_sint}

The RV12 supports a single Machine-Mode timer interrupt \texttt{EXT\_SINT}.

The interrupt may be delegated to other operating modes via software
manipulation of \texttt{mip} and \texttt{sip} registers (See sections 7.6.8 and 7.7.3).
Alternatively, higher performance interrupt redirection may be
implemented via use of the \texttt{mideleg} and \texttt{sideleg} configuration registers
(See sections \ref{machine-exception-interrupt-delegation-registers-medeleg-mideleg} and \ref{supervisor-trap-delegation-registers-sedeleg-sideleg} ).

The interrupt vector used to service the interrupt is determined based
on the mode the interrupt is delegated to via the \texttt{MTVEC\_DEFAULT},
\texttt{STVEC\_DEFAULT} and \texttt{UTVEC\_DEFAULT} parameters.

\subsubsection{EXT\_INT}\label{ext_int}

RV12 supports one general-purpose external interrupt input per operating
mode, as defined in Table \ref{tab:external-interrupt-inputs}:

\begin{longtable}[]{@{}lll@{}}
\toprule
Interrupt & Priority & Mode Supported\tabularnewline
\midrule
\endhead
EXT\_INT{[}3{]} & 3 & Machine Mode\tabularnewline
EXT\_INT{[}2{]} & 2 & Reserved\tabularnewline
EXT\_INT{[}1{]} & 1 & Supervisor Mode\tabularnewline
EXT\_INT{[}0{]} & 0 & User Mode\tabularnewline
\bottomrule
\caption{External Interrupt Inputs}
\label{tab:external-interrupt-inputs}
\end{longtable}

Each interrupt will be serviced by the operating mode it corresponds to,
or alternatively a higher priority mode depending on the system
configuration and specific operating conditions at the time the
interrupt is handled. This includes if interrupt delegation is enabled,
if a specific is implemented, or the specific operating mode at the time
of servicing for example.

Notes:

\begin{enumerate}
\def\labelenumi{\arabic{enumi}.}
\item
  An external interrupt will never be serviced by a lower priority mode
  than that corresponding to the input pin. For example, an interrupt
  presented to \texttt{EXT\_INT{[}1{]}} -- corresponding to supervisor mode --
  cannot be serviced by a user mode ISR.
\item
  Conversely, Machine Mode may service interrupts arriving on any of the
  interrupt inputs due to it have the highest priority.
\end{enumerate}